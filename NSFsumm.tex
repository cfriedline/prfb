
%%%%%%%%% SUMMARY -- 1 page, third person
% e.g:  "The PI will prove" not "I will prove"

\required{Project Summary}
% This should be a brief statement of the problem you plan to address.
% It should look something like an abstract. 

The main goal of this project is to investigate bark thickness as a fire-adapted phenotype in four, economically-important, 
species of pines with a range along the eastern and south-eastern United States.  The evolution of this complex trait will 
be studied in populations of slash pine (\textit{Pinus elliottii}), pond pine (\textit{P.\ serotina}), loblolly pine (\textit{P.\ taeda}), 
and long leaf pine (\textit{P.\ palustris}).  Next generation sequencing (NGS) of 25 individuals from 20 natural populations for each 
species (25 x 20 x 4 = 2000 total individuals) will be used to genotype individuals with the goal of uncovering shared 
genetic architecture of these four closely related species. The results of this research, while also providing new genomic 
resources for non-model species and augmenting exiting resources, will not only address fundamental principles in 
evolutionary biology, but also serve to inform and improve breeding programs and land management initiatives.

Fire has played an important role in the evolutionary paths of many adaptive traits in plants, across a wide variety of species 
and habitats.  Extant \textit{Pinaceae} species occupy a variety of habitats across North and Central America, Asia, and Europe.  
These areas have been strongly influenced by fire, and as such, their plants have developed a range 
of fire-adapted traits.  This study will investigate the underlying genetic architecture associated with the complex 
bark thickness phenotype, the degree to which this architecture is shared among four closely related species across their natural 
ranges and multiple evolutionary time scales, and how this variability can inform current economic interests.  \textbf{Thus, this research 
directly applies to the overall goal of the NPGI by developing a basic knowledge of the structures and functions of economically 
important plant genomes}.


\required{Intellectual Merit}
% This is why your project is interesting and will help further
% knowledge in the field of mathematics. 
By focusing on the dissection of an adaptive trait at both microevolutionary and macroevolutionary 
time scales, this study will address a fundamental principle in evolutionary biology: the mechanism 
associated with the vast majority of incomplete lineage sorting among populations of closely related species.  
The rise of shared phenotypes between related species has been a source of debate for many years, dividing 
into camps supporting either ancient inherited polymorphisms or multiple instances of convergent evolution.  Very recently, 
\citet{Segurel:vf} showed that the ABO blood group antigens in humans are the result of vertical inheritance 
of maintained, ancient polymorphisms and \citet{Roux:2012eb} demonstrated balancing selection around the S-locus in two 
species of  \emph{Arabidopsis}.  
In short, this study seeks to investigate the same phenomena in a quantitative, economically and 
environmentally-important, trait in four species of related plants at multiple evolutionary time scales.  Additionally, 
the genomic resources developed in this project will augment and facilitate knowledge transfer of genomic data in non-model 
species which, until recently, have been technologically out of reach.


\required{Broader Impacts}
% There are 4 kinds of broader impacts.
% 1. advance discovery and understanding while promoting teaching,
% training and learning
% 2. broaden the participation of underrepresented groups
% 3. disseminated broadly to enhance scientific and technological
% understanding
% 4. benefits of the proposed activity to society

Data will be made available for download and exploration from VCU-hosted FTP and web sites.  Also, the PI will develop and teach a new 
graduate course, Applied Ecological Genomics, to introduce and expose students to the bioinformatic skills necessary 
to effectively manage terabases of data as well as how to exploit NGS data to infer population-level, evolutionary processes.   
Finally, this will give the PI additional teaching and research experience needed to jumpstart an academic career, focusing on 
ecologically-relevant problems in an evolutionary context.

