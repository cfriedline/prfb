%%%%%%%%% 

% An abstract of your dissertation research (limited to one page);

% Plans for data management and sharing of the products of research. Proposals 
% must include a supplementary document of no more than two pages labeled “Data 
% Management Plan”. This supplement should describe how the proposal will conform 
% to NSF policy on the dissemination and sharing of research results (see AAG 
% Chapter VI.D.4), and may include: the types of data, samples, physical collections, 
% software, curriculum materials, and other materials to be produced in the course of the 
% project; the standards to be used for data and metadata format and content (where existing 
% standards are absent or deemed inadequate, this should be documented along with any 
% proposed solutions or remedies); policies for access and sharing including provisions for 
% appropriate protection of privacy, confidentiality, security, intellectual property, or other rights 
% or requirements; policies and provisions for re-use, re-distribution, and the production of 
% derivatives; and plans for archiving data, samples, and other research products, and for 
% preservation of access to them. Data management requirements and plans specific to the Directorate, 
% Office, Division, Program, or other NSF unit, relevant to a proposal are available at: 
% http://www.nsf.gov/bfa/dias/policy/dmp.jsp. If guidance specific to the program is not available, 
% then the requirements established in this section apply.
%
% Simultaneously submitted collaborative proposals and proposals that include subawards are a 
% single unified project and should include only one supplemental combined Data Management Plan, 
% regardless of the number of non-lead collaborative proposals or subawards included. Fastlane will 
% not permit submission of a proposal that is missing a Data Management Plan. Proposals for 
% supplementary support to an existing award are not required to include a Data Management Plan.
%
% A valid Data Management Plan may include only the statement that no detailed plan is needed, 
% as long as the statement is accompanied by a clear justification. Proposers who feel that the plan 
% cannot fit within the supplement limit of two pages may use part of the 15-page Project Description f
% or additional data management information. Proposers are advised that the Data Management 
% Plan may not be used to circumvent the 15-page Project Description limitation. The 
% Data Management Plan will be reviewed as an integral part of the proposal, coming under 
% Intellectual Merit or Broader Impacts or both, as appropriate for the scientific community of relevance.

\required{Dissertation Abstract}

\subsection*{Using Molecular Sequence Data to Unravel the Relationships between Bacteria 
and their Environment}

Microbial communities are recognized as major drivers of global biogeochemical processes.  However, 
the genetic diversity and composition, as well as processes leading to the origin and diversification of these 
communities in space and time, are poorly understood.  Characterization of microbial communities using 
high-throughput sequencing of 16S tags shows that OTU abundances can be approximated by a gamma distribution, 
which suggests structuring around small numbers of highly abundant OTUs and a large proportion of low abundant, 
rare, OTUs.  The current methods used to characterize how communities are structured rely on multivariate statistics, 
which operate on pair-wise distance matrices.  My analyses demonstrate that use of these methods, by reducing a 
highly-dimensional dataset (tens of samples, thousands of OTUs), result in a significant loss of information 
\citep{Friedline:2012fm}. I demonstrate that, in some cases, up to \SI{80}{\percent} of the most abundant OTUs may be removed while 
still recovering the same community relationships; this indicates these metrics are biased toward the highly abundant OTUs.  I 
will demonstrate that the observed patterns of OTU abundance detected from microbial communities can be properly 
modeled using techniques similar to those used to model the presence and absence of genes in genome 
evolution \citep{Lake:2004cy, Rivera:2004ct}.  Using simulation studies, I demonstrate that general Markov models 
in a Bayesian inference framework outperform traditional, multivariate ecological methods in recovering true community 
structure.  Applying this new methodology to Atlantic Ocean microbial communities allowed us to uncover a distance-decay 
effect, which was not revealed by the traditional methods \citep{Friedline:2012fm}.  Although the ocean dataset operated on a 
much larger, continental scale, characterization of the sequence data generated from the a nutrient poor soil on Hog Island, 
a barrier island off the Virginia Coast, allows for a better characterization of the processes affecting these communities 
on a much smaller scale.  Finally, using 16S data from the Vaginal Human 
Microbiome Project, generated here at VCU under the umbrella of the overall NIH HMP initiative, I show that 
quality filtering has a profound effect on the reliability of downstream analysis.  In conclusion, 
my analyses of the metagenomic sequence data from three types of bacterial communities demonstrate that the proper 
identification of the biological process influencing these communities requires the development and implementation of 
new statistical and computational methodology that takes advantage of the extensive amount of information generated by 
high-throughput sequencing. 

\newpage

\required{Data Management Plan}

The PI is committed to fostering an open and transparent process of data generation, 
analysis, and publication. As required in the NSF PGRP guidelines, genome level sequence data will be made 
available as soon as data quality and integrity have been verified via simple FTP sites hosted by sponsor Eckert. 
In addition, we will deposit data and computer code in standard public repositories prior to, and in all cases 
regardless of, publication (e.g., GenBank, Dryad, TreeGenes).

\subsection*{Data Types}

\paragraph{Genetic}
Genetic sequence data from the Illumina platform will be stored as compressed FASTA and/or FASTQ files 
on disk. The quality-filtered, working set of sequence data will be stored in a relational database management system 
(RDBMS) built on PostgreSQL following an existing Rivera lab data storage model currently in production 
for the Vaginal Human Microbiome Project. Reads will be tracked through a barcoding system relating 
sequence data uniquely with individuals, populations, and additional metadata, as appropriate (e.g., sampling 
location, SNP genotypes, annotation, etc.).  A copy of the processed, quality-controlled data will be extracted 
automatically on a schedule and compressed for distribution via FTP.  Additionally, a web interface will be provided for 
dynamic data exploration.

\paragraph{Spatial and phenotypic}
All metadata will be stored in the same relational database system.  This includes latitude and longitude for all sampling
locations as well as phenotypic measurements and observations (e.g., height, bark thickness, diameter a breast height (DBH)).

\subsection*{Data Standards}

All data, including data generated in downstream analysis, will be made available in a plain-text format.  Working sequence
data will be stored as text in the RDBMS, as well as associated environmental and phenotypic data.  These data will be 
made available in two ways.  First, as downloadable, compressed files available via FTP.  Second, as dynamically generated
tables available from a web interface.  Downloadable data will be in tab-delimited format, with informative headers 
as appropriate.  This format was chosen as it conforms to a standard used in programs such as R, as well as to any 
current or new custom Java and Python code developed in this project.

\subsection*{Policies for Data Access and Sharing}
Genomic data will be released following the Bermuda/Ft.\ Lauderdale agreement and will also be deposited in 
GenBank once data quality is sufficient for release. The remaining data will be released after publication in the 
text and web formats as described above, along with a copy of the publication, and will be posted on web servers in both
the Eckert and Rivera laboratories. All relevant code and schemas used to process, store, and analyze these data will 
be made available under and open source license and distributed via the web (e.g., bitbucket.org).  There will be no 
charge for these data. 

\subsection*{Provisions for Appropriate Protection and Privacy}
There are no known privacy or ethical issues associated with the data collected on this grant. 
There are no known copyright, licensing, or other intellectual property issues associated with these data.

\subsection*{Policies and Provisions for Re-use and Re-distribution}
The data will not contain any permission restrictions beyond citation of our original publication(s) where the data 
were introduced. The genetic data will be of potential interest to the landscape genomic community (both 
professional and academic) for subsequent meta analysis, statistical model development, and other summary 
uses. The PI does not foresee any reason to restrict the re-use of these data to any entity or group.

\subsection*{Plans for Archiving and Preservation of Access}
Raw DNA template will be stored at VCU in the Eckert Laboratory in 1.5ml Eppendorf tubes at \SI{-80}{\celsius} for least as 
long as it takes to publish all the materials associated with this project. Individual primers will not be made available, 
though published sequences can be used to create them de novo if necessary. Textual genetic data will be made 
available, via the web or some other publicly available medium as described above, for as long as the PI is 
engaged in academic research. Unprocessed leaf tissue samples will be similarly archived until publication, with 
raw physiological data (e.g. in the case of parameters derived from light-response curves) archived and available 
to the public following publication. For the foreseeable future, these data will be housed on the servers located within 
the Eckert Laboratory. All final data will be automatically backed up to VCU common spaces. There are no known 
issues associated with storing textual genetic data or anonymizing the identification numbers on these data sets.





















